%%%%%%%%%%%%%%%%%%%%%%%%%%%%%%%%%%%%%%%%%%%%%%%%%%%%%%%%%%%%%%%%%%%%%%%%%%%%%%%
%% University of Ulm
%%
%% Faculty of Engineering and Computer Science
%% Institute of Measurement, Control and Microtechnology
%%
%%
%% Template for Student Research Projects, Diploma, Bachelor and Master Thesis
%%
%% by order of Dr.-Ing. Michael Buchholz
%% created by Stephan Grensemann 2010
%%%%%%%%%%%%%%%%%%%%%%%%%%%%%%%%%%%%%%%%%%%%%%%%%%%%%%%%%%%%%%%%%%%%%%%%%%%%%%%

\documentclass[]{mrmthesis}     % <-- [your options (see docu)]
%defaults if no options are used: language=de, Master Thesis, confidential=false,namebehindauthortitle=false,

%new since version 1.6: option "namebehindauthortitle" for author title in front of name,
%   new default is title behind name
%Example:  \documentclass[namebehindauthortitle]{mrmthesis} % for title like "Dipl.-Ing. (FH)"

%new since version v1.6: use of biber instead of bibtex8 to generate biblatex labels (i.e. biblatex backend)
%   please set new option "backendbibtex=true" if bibtex8 is used (not recommended, not longer supported)
%   if biber is not part of your (La)TeX distribution, you can get it from http://biblatex-biber.sourceforge.net/

%new since version 1.4: option "confidential" for theses which should not be published
%Example:  \documentclass[confidential]{mrmthesis}

%2015/05/11 v. 1.9b Uni Ulm - Measurement, Control and Microtechnology
%%%%%%%%%%%%%%%%%%%%%%%%%%%%%%%%%%%%%%%%%%%%%%%%%%%%%%%%%%%%%%%%%%%%%%%%%%%%%%%
%loading extra packages / short list of recommended packages

\usepackage{fixltx2e}                  %fix problems in LaTeX that it "`would be nice"' to correct before next LaTeX release

%note: AMSmath Package is already loaded by mrmthesis to avoid problems with hyperref

%\usepackage{amssymb}                   %special characters in AMS
%\usepackage{amsfonts}                  %TeX fonts from the American Mathematical Society
%\usepackage{dsfont}                    %Doublestroke Font for set symbols, e.g. real numbers \mathds{R}
%\usepackage{amsbsy}                    %producing bold mathematics symbols where appropriate fonts exist
%\usepackage{mathrsfs}                  %Support use of the Raph Smith�s Formal Script font in mathematics
%\usepackage{mathdots}                  %Commands to produce dots in math that respect font size
%\usepackage{upgreek}                   %provides upright greek letters

%\usepackage{siunitx}                   %International System of Units
%\usepackage{textcomp}                  %provide many text symbols in the TS1 encoding

%\usepackage{xspace}                    %provides a single command that looks at what comes after it in the command stream,
                                        %and decides whether to insert a space to replace one "eaten" by the TeX command decoder

%\usepackage{float}                     %improved interface for floating objects
%\usepackage{rotating}                  %rotation tools, including rotated full-page floats

%\usepackage{ltxtable}                  %merged tabularx - longtable package
%\usepackage{booktabs}                  %publication quality tables in LaTeX

%\usepackage{listings}                  %Typeset source code listings using LaTeX
%\lstset{language=html,%
%           frameround=fttt,%
%           breaklines=true,%
%           numbers=left,%
%           numberbychapter=true,%
%           captionpos=b,%
%           basicstyle=\scriptsize}%

%%OLD:
%\usepackage{pstricks}                  %set of macros that allow the inclusion of PostScript drawings directly inside LaTeX code
%\usepackage{auto-pst-pdf}              %Package for using pstricks/PSfrag with PDF(La)TeX, requires PERL!
                           % PLEASE NOTE: to use this package you have to add
                           %  "`--enable-write18"' as compiler argument in the
                           %  LaTeX => PDF output-profile (see description in docu)
%
%IF MODIFICATION OF EPS FILES IS NECESSARY, BETTER USE THE NEW PACKAGE pstool
%\usepackage{pstools}
                           % PLEASE NOTE: to use this package you have to add
                           %  "`--enable-write18"' as compiler argument in the
                           %  LaTeX => PDF output-profile (see description in docu)
%
%THE BEST SOLUTION: USE pgfplots/tikz INSTEAD OF EPS FILES (see also tool "matlab2tikz")
%\usepackage{tikz}                      %language for producing vector graphics from a geometric/algebraic description
%\usepackage{pgfplots}                  %draws high-quality function plots in normal or logarithmic scaling

%\usepackage{epstopdf}                  %convert EPS to 'encapsulated' PDF using GhostScript
                           % PLEASE NOTE: to use this package you have to add
                           %  "`--enable-write18"' as compiler argument in the
                           %  LaTeX => PDF output-profile (see description in docu)

%\usepackage{nomencl}                   %produce lists of symbols as in nomenclature
%\ifthenelse{\boolean{en}}{}{\renewcommand{\nomname}{Nomenklatur}}
%\makenomenclature

%%%%%%%%%%%%%%%%%%%%%%%%%%%%%%%%%%%%%%%%%%%%%%%%%%%%%%%%%%%%%%%%%%%%%%%%%%%%%%%

%!!!
%Regarding the compatibility for different platforms and LATEX distributions, the mrmthesis class requires
% "latin1" (ISO 8859-1) encoding of all files. Especially UNIX user should ensure this encoding using the
% options of your preferred editor.
%!!!

%--------------------------------------------------------------
%% set options for mrmbibstyle:
%% url, ISBN, eprint, DOI: content of respective field is never printed if set to false (default)
%% printlanguage: if set to false (default), the language field is never printed. set to true and use standard
%%      biblatex option clearlang to print language fielf for specific languages (see biblatex manual)
%% yearonly: if set true (default), do not print day or month of a publication even if given
%% titlesentencecase: If set to false (default), all titles are written as given in the bib file. If set to true,
%%      all titles of articels, inproceedings etc. (but not booktitle etc.) are transformed to sentence case,
%%      i.e. only the first letter of the title is capital, all other words start with small letters even if
%%      capital letters in the bib file. Use additional curly braces e.g. for abbreviations which should
%%      remain capital letters (as in original bibtex). The command has no effect on entries where the entry
%%      field "hyphenation" is set to e.g. "ngerman".
\ExecuteBibliographyOptions{eprint=false, url=false, isbn=false, doi=false, yearonly=true, printlanguage=false, titlesentencecase=false}

%\ExecuteBibliographyOptions{firstinits=true} %activate, if all given names shall be abbreviated to initials

%%Use field hyphenation for bib entries to set the language of the bib entry, e.g. for correct hyphenation.
%%Do not mix up with the hyphenation command below!
\ExecuteBibliographyOptions{babel=hyphen} %uses hyphenation for language given in entry's hyphenation field (or global language otherwise)


%% Using new command to add bib file, set options for not printing
\addbibresource{my_mrmthesis.bib} % <- type in your bibfile

%% If the above does not work with older versions of biblatex,
%% try the compatibility mode using the folling command instead:
%\bibliography{my_mrmthesis}      % <- type in your bibfile
%--------------------------------------------------------------

%\hyphenation{}                         %insert hyphenations for unknown words


%Datei f�r eigene Kommandos, z.B.


%Typographie: (ben�tigen Package xspace)
%\newcommand{\dasheisst}{\mbox{d.\,h.}\xspace}
%\newcommand{\unteranderem}{\mbox{u.\,a.}\xspace}
%\newcommand{\zumbeispiel}{\mbox{z.\,B.}\xspace}

%Befehl f�r deutsche Anf�hrungszeichen: (ben�tigt babel mit (n)german)
%\shorthandon{"} %bei Einbindung der Command-Datei vor \begin{document} n�tig
%\newcommand{\gq}[1]{"`#1"'} %doppelte Anf�hrungszeichen, Verwendung \gq{in Anf�hrungszeichen}
%\newcommand{\sgq}[1]{\glq #1\grq{}} %einfache Anf�hrungszeichen, Verwendung \sgq{in Anf�hrungszeichen}
%\shorthandoff{"} %bei Einbindung der Command-Datei vor \begin{document} n�tig


%Mathematisches
%%%%%%%%%%%%%%%
%
%Befehl \vm zur Notation von Vektoren und Matrizen in Fettschrift
%\newcommand{\vm}[1]{\protect\ensuremath\boldsymbol{#1}} %ben�tigt Paket amsbsy
%Befehl \vm zur Notation von Vektoren und Matrizen mit Unterstrich:
%\newcommand{\vm}[1]{\protect\ensuremath\underline{#1}}

%Befehl \trans f�r korrekt gesetztes "T" im Mathemodus f�r "Transponiert
%\newcommand{\trans}{^\text{\rm \textup{T}}} %Verwendung: \vm A\trans
%�hnliche Befehle:
%\newcommand{\inv}{^{-1}} %Inverse
%\newcommand{\pinv}{^{\dag}} %Pseudoinverse
%\newcommand{\opt}{\ensuremath ^\star} %Optimale L�sung

%angepasste Versionen f�r optisch besseren Satz bei einigen Buchstaben (z.B. Y,A):
%\newcommand{\transnah}{^{\!\text{\rm \textup{T}}}}
%\newcommand{\invnah}{^{\!-1}}
%\newcommand{\pinvnah}{^{\!\dag}}
%\newcommand{\optnah}{\ensuremath ^{\!\star}}


%aufrechtes "e" f�r die e-Funktion, als Argument die Potenz:
%\newcommand{\efkt}[1]{\ensuremath {{\rm e}}^{#1}}

%Formeln im Text ohne Zeilenumbruch:
%\newcommand{\teq}[1]{\mbox{$#1$}}

%Befehl f�r PT_n-Glied:
%\newcommand{\PT}[1]{\teq{\text{PT}_{#1}}}

%Einheitsmatrix:
%\newcommand{\eye}[1]{\vm{I}}
%mit Dimension:
%%\newcommand{\eye}[1]{\vm{I}_{\!#1}}

%weitere mathematische Funktionen:
%\DeclareMathOperator{\trace}{spur}
%\DeclareMathOperator{\sign}{sign}
%\DeclareMathOperator{\diag}{diag}


%Normen:
%\newcommand{\frobnorm}[1]{\ensuremath \left\| #1 \right\|_\text{F}}
%\newcommand{\zweinorm}[1]{\ensuremath \left\| #1 \right\|_2}
%\newcommand{\vmnorm}[1]{\ensuremath \left\| #1 \right\|}
                   %insert file with self-defined commands. The file exists and contains many examples.

%%%%%%%%%%%%%%%%%%%%%%%%%%%%%%%%%%%%%%%%%%%%%%%%%%%%%%%%%%%%%%%%%%%%%%%%%%%%%%%
%please fill out the following required information
\title{Title of your thesis}
%\descriptiontitle{title for thesis description page with different line-breaks, if necessary} % Default: \title
%\affirmationtitle{title for affirmation page with different line-breaks, if necessary} % Default: \title
\author[m]{Your Name} %use optional argument [f] for female label, default is [m] for male
%\authortitle{B.\,Sc.} %default: None, please note: use class option namebehindauthortitle if title should be printed in front of the name
\supervisor[m]{Dr.-Ing. Mike Slackenerny} %use optional argument [f] for female label, default is [m] for male
\examiner{Prof. Dr.-Ing. Brian F. Smith}
\coexaminer{Prof. Rivera}
\issuedate{01.01.2001}
\submissiondate{31.12.2001}
%\place{}                       %default: Ulm

%%%%since v. 1.4: Possible re-definition of phrases for "confidential"-option
%%%%please note: Normally, there should be no reason to change these phrases
%%%% and changing can cause unwanted behaviour.
%%%\confidentialMainText{}
%%%\confidentialSubText{}

%%%%%%%%%%%%%%%%%%%%%%%%%%%%%%%%%%%%%%%%%%%%%%%%%%%%%%%%%%%%%%%%%%%%%%%%%%%%%%%
%main text
\begin{document}
% - - - - - - - - - - - - - - -
 \frontmatter
\maketitle
%  \projectdescription{\input{doc/projectdescription}}   % the file "projectdescription.tex" should be provided by your supervisor
  \affirmation

  %\extrafrontchapter{Foreword}{type in your text here} %or some other optional stuff
  %\extrafrontchapter{Thanks}{type in your text here}

  \tableofcontents
  %\listoffigures
  %\listoftables
  %\printnomenclature        %PLEASE NOTE: If you don't change the MakeIndex argument
                             % in the output-profile into
                             % "%bm.nlo" -s nomencl.ist -o "%bm.nls", no output will be produced
% - - - - - - - - - - - - - - -
 \mainmatter                       %import chapters (tip: use one tex-file for each chapter)
   %\input{doc/}
   %\input{doc/}
   %\input{doc/}
   %\input{doc/}
   %\input{doc/}
% - - - - - - - - - - - - - - -
  \appendix                        %import all your appendix stuff here
   %\input{doc/}
   %\input{doc/}
% - - - - - - - - - - - - - - -
  \backmatter
%  \nocite{*} %add all items of bib file to bibliography. Replace "*" by a list of specific
%             % bibentry keys to select only some, or comment this line
%             %normally, all bib entries should be cited in the text
   \printbibliography[heading=bibintoc]
% - - - - - - - - - - - - - - -
\end{document}                      %yeah, you're done
