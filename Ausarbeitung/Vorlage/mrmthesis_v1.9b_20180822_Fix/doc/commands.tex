%Datei f�r eigene Kommandos, z.B.


%Typographie: (ben�tigen Package xspace)
%\newcommand{\dasheisst}{\mbox{d.\,h.}\xspace}
%\newcommand{\unteranderem}{\mbox{u.\,a.}\xspace}
%\newcommand{\zumbeispiel}{\mbox{z.\,B.}\xspace}

%Befehl f�r deutsche Anf�hrungszeichen: (ben�tigt babel mit (n)german)
%\shorthandon{"} %bei Einbindung der Command-Datei vor \begin{document} n�tig
%\newcommand{\gq}[1]{"`#1"'} %doppelte Anf�hrungszeichen, Verwendung \gq{in Anf�hrungszeichen}
%\newcommand{\sgq}[1]{\glq #1\grq{}} %einfache Anf�hrungszeichen, Verwendung \sgq{in Anf�hrungszeichen}
%\shorthandoff{"} %bei Einbindung der Command-Datei vor \begin{document} n�tig


%Mathematisches
%%%%%%%%%%%%%%%
%
%Befehl \vm zur Notation von Vektoren und Matrizen in Fettschrift
%\newcommand{\vm}[1]{\protect\ensuremath\boldsymbol{#1}} %ben�tigt Paket amsbsy
%Befehl \vm zur Notation von Vektoren und Matrizen mit Unterstrich:
%\newcommand{\vm}[1]{\protect\ensuremath\underline{#1}}

%Befehl \trans f�r korrekt gesetztes "T" im Mathemodus f�r "Transponiert
%\newcommand{\trans}{^\text{\rm \textup{T}}} %Verwendung: \vm A\trans
%�hnliche Befehle:
%\newcommand{\inv}{^{-1}} %Inverse
%\newcommand{\pinv}{^{\dag}} %Pseudoinverse
%\newcommand{\opt}{\ensuremath ^\star} %Optimale L�sung

%angepasste Versionen f�r optisch besseren Satz bei einigen Buchstaben (z.B. Y,A):
%\newcommand{\transnah}{^{\!\text{\rm \textup{T}}}}
%\newcommand{\invnah}{^{\!-1}}
%\newcommand{\pinvnah}{^{\!\dag}}
%\newcommand{\optnah}{\ensuremath ^{\!\star}}


%aufrechtes "e" f�r die e-Funktion, als Argument die Potenz:
%\newcommand{\efkt}[1]{\ensuremath {{\rm e}}^{#1}}

%Formeln im Text ohne Zeilenumbruch:
%\newcommand{\teq}[1]{\mbox{$#1$}}

%Befehl f�r PT_n-Glied:
%\newcommand{\PT}[1]{\teq{\text{PT}_{#1}}}

%Einheitsmatrix:
%\newcommand{\eye}[1]{\vm{I}}
%mit Dimension:
%%\newcommand{\eye}[1]{\vm{I}_{\!#1}}

%weitere mathematische Funktionen:
%\DeclareMathOperator{\trace}{spur}
%\DeclareMathOperator{\sign}{sign}
%\DeclareMathOperator{\diag}{diag}


%Normen:
%\newcommand{\frobnorm}[1]{\ensuremath \left\| #1 \right\|_\text{F}}
%\newcommand{\zweinorm}[1]{\ensuremath \left\| #1 \right\|_2}
%\newcommand{\vmnorm}[1]{\ensuremath \left\| #1 \right\|}
